\chapter{Resultados}

Neste cap�tulo ser�o apresentados os resultados obtidos atrav�s das t�cnicas propostas. Para isso ser�o utilizados quatro conjuntos de sinais distintos, dois de dados simulados, com diferentes cortes de n�vel 1 e outros dois com sinais medidos no detector.

\section{Resultados - Dados simulados}

No trabalho \cite{article:simas:2008:eusipco}
foi proposto um crit�rio para a escolha do n�mero de agrupamentos em
um problema ``cego" de processamento de sinais.


\section{Resultados - Dados experimentais}



\section{Estudo a Respeito do Tempo de Processamento dos Algoritmos}

Neste cap�tulo ser� apresentado um estudo a respeito do tempo de processamento exigido pelos algoritmos 
propostos. Considerando que a aplica��o exige uma resposta em no m�ximo 40 ms, o tempo de processamento 
� um fator limitante na escolha dos algoritmos de filtragem.

\subsection{Rotinas comuns a todos os algoritmos}

O algoritmo anelador (respons�vel pela gera��o dos sinais em an�is) foi implementado como um ... do T2Calo 
(discriminador oficial do ATLAS) e portanto utiliza parte das informa��es produzidas por esse �ltimo. 